\hypertarget{_pixel_8hpp}{
\subsection{Pixel.hpp File Reference}
\label{_pixel_8hpp}\index{Pixel.hpp(145)@{Pixel.hpp(145)}}
}


\subsubsection{Detailed Description}


\subsubsection*{Typedefs}
\begin{CompactItemize}
\item 
typedef std::pair$<$ unsigned int, unsigned int $>$ \hyperlink{_pixel_8hpp_535e59456e3e633842529cfa8ea103c4}{Pixel}
\begin{CompactList}\small\item\em Coordinates of an image pixel. \item\end{CompactList}\end{CompactItemize}


\subsubsection{Typedef Documentation}
\hypertarget{_pixel_8hpp_535e59456e3e633842529cfa8ea103c4}{
\index{Pixel.hpp@{Pixel.hpp}!Pixel@{Pixel}}
\index{Pixel@{Pixel}!Pixel.hpp@{Pixel.hpp}}
\paragraph[Pixel]{\setlength{\rightskip}{0pt plus 5cm}{\bf Pixel}}\hfill}
\label{_pixel_8hpp_535e59456e3e633842529cfa8ea103c4}


This pair keeps the coordinates of a pixel in a press clip. The first member representes the x coordinate (the row) and the second member representes the y coordinate (the column)

\begin{Desc}
\item[Author:]Eliezer Talón (\href{mailto:elitalon@gmail.com}{\tt elitalon@gmail.com}) \end{Desc}
\begin{Desc}
\item[Date:]2008-10-13 \end{Desc}
